\subsection{Adaptação das imagens do jogo}

Esta fase foi marcada principalmente pelo entendimento de como imagens são carregadas e interpretadas pelo GBA.

Primeiramente, era necessário conhecer a resolução das imagens no jogo original e no jogo a ser portado para que pudesse ser estabelecida uma proporção a ser utilizada na adaptação das imagens. Para isso, foram utilizadas as imagens de \textit{background}, pois elas ocupam todo o espaço da janela do jogo original. A altura dos \textit{backgrounds} no jogo original é \textit{480px} e, sabendo que a tela do GBA possui \textit{160px}, é possível estabelecer um fator de conversão bastante preciso, seguindo a fórmula

\begin{equation}
\label{Cálculo da proporção das imagens do jogo}
\frac{480_{px}}{160_{px}} = \frac{3}{1}
\end{equation}

Portanto, a proporção das imagens do jogo original para o jogo a ser portado é de \texttt{3:1}.

Após realizar o redimensionamento das imagens para o tamanho correto, foi necessário descobrir como utilizar essas imagens no GBA. Diferentemente de sistemas mais modernos, o GBA não carrega imagens de fato, e sim um código C que contém as informações da imagem, como paleta de cores, \textit{tiles} e o mapeamento desses \textit{tiles} na imagem. Para realizar a conversão da imagem para este código, foi utilizada a ferramenta GRIT (\textit{GBA Raster Image Transmogrifier}). Com essa ferramenta é possível converter a imagem utilizando uma série de parâmetros, como quantidade de bits por pixel da imagem, formato de redução de \textit{tiles}, formato de saída do arquivo gerado (C, \textit{Assembly}, entre outros), altura e largura, em \textit{tiles} (conjuntos de \textit{8px} x \textit{8px}), da imagem , dentre outras opções. Para a conversão das \textit{sprites} do jogo, o seguinte comando foi utilizado:

\begin{lstlisting}[language=bash,caption={Comando para conversão das imagens em código}]
$ grit nome-da-imagem.png -gB4 -ftc -Mw2 -Mh4
\end{lstlisting}

Nesse comando, o código relativo à imagem é gerado considerando que cada \textit{pixel} da imagem utiliza 4 \textit{bits} (\textbf{-gb4}), exportando a imagem como código C (\textbf{-ftc}) e definindo a largura e a altura da imagem como sendo, respectivamente, \textit{16px} (\textbf{-Mw2}) e \textit{32px} (\textbf{-Mh4}). O resultado da execução dessa instrução é um \textit{header} e um código-fonte correspondentes à imagem, contendo uma paleta com as cores utilizadas e um vetor de \textit{tiles}, assim como mostrado nos códigos \ref{lst:imageheader} e \ref{lst:imagecpp}.


\begin{lstlisting}[caption={Cabeçalho da parte superior da imagem da plataforma da primeira fase.},label={lst:imageheader}]
//{{BLOCK(level1_plat0)

//======================================================================
//
//  level1_plat0, 16x32@4,
//  Transparent palette entry: 17.
//  + palette 16 entries, not compressed
//  + 8 tiles Metatiled by 2x4 not compressed
//  Total size: 32 + 256 = 288
//
//  Time-stamp: 2018-11-28, 22:18:37
//  Exported by Cearn's GBA Image Transmogrifier, v0.8.15
//  ( http://www.coranac.com/projects/#grit )
//
//======================================================================

#ifndef GRIT_LEVEL1_PLAT0_H
#define GRIT_LEVEL1_PLAT0_H

#define level1_plat0TilesLen 256
extern const unsigned int level1_plat0Tiles[64];

#define level1_plat0PalLen 32
extern const unsigned short level1_plat0Pal[16];

#endif // GRIT_LEVEL1_PLAT0_H

//}}BLOCK(level1_plat0)
\end{lstlisting}

\begin{lstlisting}[caption={\textit{Código fonte} da parte superior da imagem da plataforma da primeira fase.},label={lst:imagecpp}]
//{{BLOCK(level1_plat0)

//======================================================================
//
//  level1_plat0, 16x32@4,
//  Transparent palette entry: 17.
//  + palette 16 entries, not compressed
//  + 8 tiles Metatiled by 2x4 not compressed
//  Total size: 32 + 256 = 288
//
//  Time-stamp: 2018-11-28, 22:18:37
//  Exported by Cearn's GBA Image Transmogrifier, v0.8.15
//  ( http://www.coranac.com/projects/#grit )
//
//======================================================================

const unsigned int level1_plat0Tiles[64] __attribute__((aligned(4))) __attribute__((visibility("hidden")))=
{
    0xCCCCCCCC,0xAAACCCCC,0x888AACCB,0x89988BBA,0x99889888,0x99898221,0x98822222,0x81122444,
    0xCCCCCCCC,0xCCCCCCBA,0xCCCCCA89,0xBB888989,0x78888988,0x25889988,0x44225889,0x22444168,
    0x22244223,0x44444444,0x44444444,0x44444444,0x44444444,0x44444444,0x44444444,0x44444444,
    0x44223432,0x44444244,0x44444444,0x44444444,0x44444444,0x44444444,0x44444444,0x44444444,
    0x44444444,0x44444444,0x44444444,0x43344334,0x24422222,0x42222222,0x22222222,0x22222222,
    0x44444444,0x44444444,0x44444444,0x33444344,0x44223432,0x22444244,0x22222222,0x11222222,
    0x22222222,0x21122222,0x11111111,0x11111111,0x11111111,0x11111111,0x11111111,0x11111111,
    0x11222222,0x11111122,0x11111111,0x11111111,0x11111111,0x11111111,0x11111111,0x11111111,
};

const unsigned short level1_plat0Pal[16] __attribute__((aligned(4))) __attribute__((visibility("hidden")))=
{
    0x0002,0x0889,0x088B,0x088D,0x10D0,0x092D,0x09AD,0x0A50,
    0x0A90,0x1290,0x0AD0,0x1332,0x1B74,0x0000,0x0000,0x0000,
};

//}}BLOCK(level1_plat0)
\end{lstlisting}

Já para a conversão das imagens dos \textit{backgrounds} foi utilizado o seguinte comando:

\begin{lstlisting}[language=bash,caption={Comando para conversão dos \textit{backgrounds} em código}]
$ grit nome-da-imagem.png -gB4 -ftc -mRtf -mp0
\end{lstlisting}

Assim como na imagem anterior, o código é gerado em C considerando que cada \textit{pixel} da imagem utiliza 4 \textit{bits}. Além disso, o código é gerado utilizando redução completa de \textit{tiles} (\textbf{-mRtf}) e, diferentemente das sprites, onde o índice a ser utilizado na paleta de cores é especificado apenas no código, é necessário especificar, já nessa instrução (\textbf{-mp0}), o índice da paleta de cores que será utilizada no código para guardar as cores da imagem. Vale ressaltar que isso é necessário apenas caso a imagem esteja sendo convertida a 4 \textit{bits} por \textit{pixel}, já que a 8 \textit{bits} por \textit{pixel} é possível utilizar apenas uma única paleta de cores, assim como já explicado na seção \ref{gba-memory}.

Nesse ponto da explicação é necessário saber diferenciar 3 coisas: o \textit{background} do GBA, as imagens que irão compor o \textit{background} do jogo e o que de fato vai ser mostrado na tela. O \textit{background} do GBA é todo o espaço disponível para o mapa do jogo, ele determina os limites do jogo. Se ele tiver \textit{512px} de largura, os limites horizontais do mapa irão de 0 a 512. Esse será o espaço disponível para carregamento de todo e qualquer elemento do jogo, seja ele \textit{sprite} ou imagem de fundo. Porém, nem tudo que estiver carregado no \textit{background} do GBA será necessariamente mostrado na tela, afinal a tela do GBA possui apenas \textit{240px} de largura e \textit{160px} de altura. O que será de fato mostrado dependerá do que será explicado à frente.

Nos modos de vídeo baseados em \textit{tiles}, o GBA possui um total de 4 \textit{backgrounds}, sendo que existem restrições explícitas para o tamanho de cada um deles. As dimensões aceitas são as seguintes: 256x256, 512x256, 256x512, 512x512. Cada um desses \textit{backgrounds} possui um registrador por meio do qual é possível escolher suas dimensões. A escolha das dimensões é feita por meio de uma \textit{flag} atribuída diretamente no registrador \texttt{REG\_BGxCNT}. Cada um desses \textit{backgrounds} pode possuir um tamanho diferente, desde que siga as restrições já citadas.

Para passar a ideia de que o personagem está se movimentando durante o jogo, as imagens dos \textit{backgrounds} se movem horizontalmente em velocidades diferentes, técnica chamada \textit{parallax}. Fazer isso convertendo a imagem em uma maior, a replicando em um editor de imagens para posteriormente carregá-la por completo no código provavelmente funcionaria em uma plataforma atual. Porém, no GBA, onde a memória é extremamente limitada, não seria possível nem mesmo carregar tal imagem e mesmo em uma plataforma atual muita memória seria gasta desnecessariamente. Por isso, para realizar essa movimentação, os \textit{backgrounds} do jogo são reproduzidos mudando o ponto onde a renderização começa, sendo que sempre que o \textit{background} chega ao fim, a renderização continua do início. E aqui é importante ressaltar que é o fim do \textit{background} do GBA e não das imagens, por isso é importante que as imagens possuam os mesmos tamanhos dos \textit{backgrounds} em que estão sendo renderizadas nos eixos aonde a movimentação ocorre. Por exemplo, em um \textit{background} de 512x256, em que a imagem de fundo se move somente no sentido horizontal, é necessário que essa imagem de fundo possua \textit{512px} de largura, porém para a altura, basta que a imagem possua a altura da tela, já que ela não será movimentada na vertical e por isso os \textit{pixels} da parte de cima nunca serão vistos.

Na versão original do jogo, o movimento dos \textit{backgrounds} também é realizado dessa forma, porém tendo que manualmente (no código), manipular os índices onde a renderização inicia para passar a ideia de que o \textit{background} é infinito. O GBA facilita esse processo, pois possui registradores que sinalizam o ponto onde acontece o início da renderização da imagem e quando a imagem chega ao fim, ele automaticamente emenda o fim com o início. Dessa forma, se a imagem possuir \textit{256px} de largura, basta que o valor contido no registrador seja continuamente incrementado, sem que seja necessário resetar o valor quando a imagem chegar no seu limite. Outra vantagem que o GBA proporciona ao realizar essa movimentação dos \textit{backgrounds} é o fato de que não há tamanho mínimo para a interseção entre o fim e o início da imagem. Na versão original, por exemplo, em que a resolução era de \textit{852px x 480px}, era necessário que os \textit{852px} finais da imagem fossem iguais aos \textit{852px} iniciais, pois assim, quando o índice do fundo fosse resetado, isso não seria vísivel para quem está jogando, passando a ideia de continuidade. Esse fator facilitou até mesmo a edição das imagens, fazendo com que o \textit{background} pudesse ser mais diverso, sem que fosse necessário gastar grande parte do fim da imagem apenas replicando o início. No código \ref{lst:update_bg}, é possível visualizar uma amostra de como é feita a atualização dos índices. É importante ressaltar que esse efeito da renderização continuar no início acontece para qualquer textura, e não apenas com os mapas. Sendo assim, uma \textit{sprite} cujas extremidades ultrapassem algum dos limites do \textit{background} aonde ela está sendo renderizada (o mais à frente) terá a parte que ultrapassou o \textit{background} renderizada no lado oposto.

\begin{lstlisting}[caption={Método responsável por atualizar os índices de renderização do \textit{background}},label={lst:update_bg}]
void Background::update_self(uint64_t dt) {
    if (dt % frames_to_skip == 0) {
        m_x += m_speed_x;
        m_y += m_speed_y;
    }   

    switch(this->background_id) {
        case 0:
            REG_BG0HOFS = m_x;
            REG_BG0VOFS = m_y;
            break;
        case 1:
            REG_BG1HOFS = m_x;
            REG_BG1VOFS = m_y;
            break;
        case 2:
            REG_BG2HOFS = m_x;
            REG_BG2VOFS = m_y;
            break;
        case 3:
            REG_BG3HOFS = m_x;
            REG_BG3VOFS = m_y;
            break;
        default:
            print("Invalid background id\n");
            break;
    }   
}
\end{lstlisting}

Dessa forma, sempre que a variável \texttt{m\_x}, que representa o ponto no eixo X onde começa a renderização do \textit{backgrounds} na tela, é atualizado, o registrador correspondente também é atualizado. O mesmo se aplica à variável \texttt{m\_y}, que representa o ponto no eixo Y.
 
Na versão original do jogo, as imagens das plataformas possuíam \textit{20px} de largura e \textit {400px} de altura. Porém, o GBA possui explícitas limitações em relação às dimensões das imagens a serem renderizadas. Sendo assim, é necessário que as imagens estejam em uma das dimensões mostradas na tabela \ref{table:sprite-sizes}. Tendo em vista que a altura máxima de uma \textit{sprite} no GBA é \textit{64px}, foi necessário dividir a imagem em várias para que fosse possível atingir a altura necessária das plataformas no jogo. As dimensões escolhidas para essas novas imagens foram \textit{16px} de largura por \textit{32px} de altura, pois assim a largura ficava mais próxima da largura original, preservando ao máximo o \textit{level design} original. Já a altura de \textit{32px} foi escolhida tendo em vista que é a altura máxima que pode ser utilizada com largura \textit{16px}, como mostrado na tabela \ref{table:sprite-sizes}.

\begin{table}[htb]
\center
\begin{tabular}{|l|l|l|l|}
\hline
8x8  & 16x16 & 32x32 & 64x64 \\ \hline
16x8 & 32x8  & 32x16 & 64x32 \\ \hline
8x16 & 8x32  & 16x32 & 32x64 \\ \hline
\end{tabular}
\caption{Dimensões válidas para renderização de \textit{sprites} no GBA}
\label{table:sprite-sizes}
\end{table}

Devido a essa adaptação das plataformas, a classe \texttt{TWPlatform} possui um número variável de texturas diferentemente das classes que representam os demais objetos do jogo, que possuem uma única textura. As plataformas podem utilizar uma textura (caso das plataformas que representam o chão, pois apenas a parte mais superior das plataformas é mostrada) ou por 4 texturas, podendo atingir uma altura máxima de \textit{128px}, correspondendo a 80\% da altura da tela do GBA. O código \ref{lst:platform-textures} apresenta o trecho responsável por carregar as texturas e o código \ref{lst:platform-set-y} o trecho responsável por calcular as alturas de cada uma das texturas.

\begin{lstlisting}[caption={\texttt{std::vector} com as texturas das plataformas sendo preenchido.},label={lst:platform-textures}]
m_textures.push_back(new Texture(1, level1_plat0Pal, level1_plat0PalLen, level1_plat0Tiles, level1_plat0TilesLen, _4BPP));
if (not m_is_floor)
{
    m_textures.push_back(new Texture(1, level1_plat1Pal, level1_plat1PalLen, level1_plat1Tiles, level1_plat1TilesLen, _4BPP));
    m_textures.push_back(new Texture(1, level1_plat2Pal, level1_plat2PalLen, level1_plat2Tiles, level1_plat2TilesLen, _4BPP));
    m_textures.push_back(new Texture(1, level1_plat3Pal, level1_plat3PalLen, level1_plat3Tiles, level1_plat3TilesLen, _4BPP));
}
\end{lstlisting}

\begin{lstlisting}[caption={Cálculo das alturas das texturas utilizadas nas plataformas},label={lst:platform-set-y}]
void TWPlatform::set_y(int y) {
    m_y = y;

    for (size_t i = 0; i < m_textures.size(); i++) {
        m_textures[i]->metadata.y = m_y + platform_height * i;

        if (m_y + platform_height * (i + 1) >= 256) {
            // hide sprite
            m_textures[i]->metadata.om = 2;
        } else {
            m_textures[i]->metadata.om = 0;
        }
    }
}
\end{lstlisting}

No código \ref{lst:platform-textures}, uma única textura é carregada caso a plataforma faça parte do chão do jogo e 4 plataformas são carregadas caso contrário. A variável \texttt{m\_is\_floor} indica se a plataforma faz ou não parte do chão. 

No código \ref{lst:platform-set-y}, é possível ver que a variável \texttt{m\_y} representa o ponto \texttt{y} (ponto onde a plataforma começa a ser renderizada na tela) da plataforma em si e por isso é utilizado para calcular o \texttt{y} de cada uma das texturas utilizadas na classe.

Ao falar sobre a adaptação do \textit{background}, foi explicado que o GBA automaticamente emenda o fim e o início da tela de fundo, a medida que o ponto onde a renderização da imagem é avançado. O mesmo funciona com as demais imagens do jogo. Sempre que uma imagem ultrapassa as dimensões do \textit{background}, ela começa a ser renderizada no lado oposto. Por exemplo, levando em consideração uma altura de \textit{256px} para o \textit{background} e lembrando que a altura da tela do GBA é de \textit{160px}, se uma imagem de \textit{32px} for renderizada no ponto \texttt{y = 150} na tela, os \textit{22px} restantes serão renderizados a partir do topo do \textit{background}, o que não necessariamente significa que esses \textit{pixels} remanescentes serão mostrados na tela, isso pelo fato de serem carregados na tela apenas os \textit{160px} finais da imagem do \textit{background}. Isso se torna um problema, porém, quando o número de \textit{pixels} que ultrapassam os limites inferiores da tela do GBA são suficientes para alcançar os \textit{160px} finais do \textit{background} e é para isso que serve o trecho que vai da linha 7 à linha 12 no código \ref{lst:platform-set-y}. Este trecho esconde a textura caso ela ultrapasse o limite inferior do \textit{background} (independente de chegar ou não ao ponto do restante da imagem ser renderizado na parte de cima da tela) e mostra caso contrário.
