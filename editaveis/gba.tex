\section{Game Boy Advance}

\subsection{Memória}

O \textit{Game Boy Advance} possui várias semelhanças com um PC como por exemplo processador, memória \textit{RAM}, ferramentas para entrada de dados e placa mãe \cite{harbour}. Ao desenvolver jogos para ambas as plataformas, porém, há uma nítida diferença em se tratando do controle do hardware por parte do programador. Ao programar em um PC, o sistema operacional fornece uma série de funções que facilitam o acesso ao hardware, enquanto no GBA, o acesso ao hardware é feito acessando diretamente uma determinada posição de memória(registrador) \cite{harbour}.

O \textit{GBATek} mapeia a memória utilizável em memória interna geral, memória interna do display e memória externa. A memória interna geral é dividida em \textit{System ROM} (BIOS), \textit{On-Board Work RAM}, \textit{On-chip Work RAM} e \textit{I/O Registers}. Já a memória interna do display divide-se em \textit{Pallete RAM}, \textit{Video RAM} (VRAM) e \textit{OBJ Attributes Memory} (OAM). Por fim, de acordo com o \textit{GBATek}, a memória externa é formada por 4 \textit{Game PAK ROM's} e uma \textit{Game PAK SRAM} \cite{harbour}. Por sua vez, o \textit{CowBite Virtual Especifications} trata essa última região de memória de forma mais geral, como \textit{Cart RAM}, e explica que ela pode ser utilizada como SRAM, \textit{Flash ROM} ou EEPROM \cite{cowbite}.

A \textit{System ROM} possui 16 KBytes de tamanho e contém a BIOS do sistema. Essa região de memória pode ser usada somente para escrita e qualquer tentativa de leitura resultará em falha \cite{cowbite}.

A \textit{On-Board Work RAM}, citada pelo \textit{GBATek}, é tratada como \textit{External Work RAM} (EWRAM) pelo \textit{CowBite Virtual Specifications}. Ela possui 256 KBytes de tamanho e é utilizada para inserir código e dados do jogo. Se um cabo \textit{multiboot} estiver presente quando o console for iniciado, a BIOS irá detectá-lo e automaticamente deverá transferir o código binário para essa região \cite{cowbite}. 

A \textit{On-Chip Work RAM} é tratada pelo o \textit{Cowbite Virtual Specifications} como \textit{Internal Work RAM} (IWRAM) e possui 32 KBytes de espaço. Dentre as RAM's do GBA, essa é a mais rápida. Levando em consideração que seu barramento possui 32 \textit{bits} de tamanho, enquanto o da \textit{System ROM} e da EWRAM possuem apenas 16 \textit{bits}, é recomendado que o código ARM de 32 \textit{bits} seja utilizado aqui, deixando o código \textit{THUMB} para ser utilizado na \textit{System ROM} e EWRAM \cite{cowbite}

A \textit{I/O RAM}, citada anteriormente como \textit{I/O Registers}, possui 1 KByte de extensão e é utilizada para acesso direto à memória, controle dos gráficos, do áudio e de outras funções do GBA \cite{cowbite}.

A \textit{Pallete RAM} possui 1 KByte de tamanho e tem como função armazenar as cores de 16 \textit{bits} necessárias quando se deseja utilizar paletas de cores. Ela possui duas áreas: uma para \textit{backgrounds} e outra para \textit{sprites}. Cada uma dessas áreas pode ser utilizada como uma única paleta de cores ou como 16 paletas de 16 cores cada \cite{cowbite}.

A \textit{Video RAM} (VRAM) possui 96 KBytes de espaço e é onde devem ser armazenados os dados gráficos do jogo para que possam ser mostrados na tela do GBA \cite{cowbite}.

A \textit{OBJ Attributes Memory} (OAM) possui 1 KByte de tamanho e é utilizada para controlar as \textit{sprites} do GBA \cite{cowbite}.

\subsection{Renderização de Vídeo}

O \textit{Game Boy Advance} possui uma série de modos de vídeo a serem escolhidos, três deles baseados em \textit{tiles} e três deles baseados em \textit{bitmaps} \cite{harbour}. A diferença entre os modos baseados em \textit{tiles} está na quantidade de \textit{backgrounds} que podem ser utilizados em cada um dos três modos e nas operações (rotação / \textit{zoom}) que podem ser aplicadas ou não em cada um deles \cite{harbour}. Já nos modos baseados em \textit{bitmaps}, as principais diferenças estão no fato dees utilizarem ou não paleta de cores, no número de \textit{bits} utilizados para representar as cores e na resolução \textit{harbour}.

Já com relação ao texto, uma das maneiras de renderizá-lo na tela do GBA é utilizando um dos modos baseados em \textit{bitmap}. Para isso, basta criar um arquivo contendo um vetor de \textit{bitmap} representando um dos caracteres do alfabeto e depois criar uma função de renderização \cite{harbour}.

\subsection{Tratamento de Input}

O GBA possui 4 teclas direcionais e 6 botões, cujos estados podem ser acessados por meio dos 10 primeiros \textit{bits} do registrador localizado no endereço 0x4000130 \cite{gbatek}. Há ainda um outro registrador, no endereço 0x4000132, que permite escolher quais pressionamentos de teclas geram interrupções \cite{cowbite}.

\subsection{Tratamento de Áudio}

O GBA fornece 4 canais de áudio utilizados para reproduzir tons e ruídos \cite{gbatek}. Apesar dos 4 canais de áudio, o console não possui um \textit{mixer} embutido, o que faz com que os programadores precisem escrever seus próprios \textit{mixers} ou utilizar bibliotecas de terceiros para tal propósito \cite{harbour}. Sem um \textit{mixer}, apenas um áudio pode ser tocado por vez, o que é chamado de reprodução assíncrona \cite{harbour}.

O primeiro canal de áudio do GBA é responsável pelo tom e pelo \textit{sweep}, ele possui um registrador para controle do \textit{sweep}, um para controle da frequência, que também permite reiniciar o áudio que está sendo tocado e um registrador para controlar o volume do áudio, o padrão de onda, o \textit{envelope Step-Time} e o \textit{envelope Direction} \cite{gbatek}.

O segundo canal de áudio funciona de forma similar ao primeiro e também é responsável pelo controle do tom. Ele não possui, porém, um registrador para controle do \textit{sweep} ou do \textit{tone envelope} \cite{gbatek}.

O terceiro canal de áudio é responsável pela saída de onda e pode ser utilizado para reproduzir áudio digital. Ele também pode ser utilizado para reproduzir tons normais a depender da configuração dos registradores. O canal em questão possui um registrador para controle da RAM de onda, um para controle do comprimento e volume do áudio e um para controle da frequência, permitindo também reiniciar o áudio que está sendo tocado.

Por fim, o quarto canal é responsável pelo ruído. Ele pode ser utilizado para reproduzir ruído branco, o que pode ser feito alternando randomicamente a amplitude entre alta e baixa em uma dada frequência. Também é possível modificar a função do gerador randômico de tal forma que a saída de áudio se torne mais regular, o que fornece uma capacidade limitada de reproduzir tons ao invés de ruído. Esse canal possui um registrador para controlar o volume do áudio, o \textit{Envelope Step-Time} e o \textit{Envelope Direction} e um registrador para controle da frequência, permitindo também reiniciar o áudio.
