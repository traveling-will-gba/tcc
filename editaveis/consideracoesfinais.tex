\chapter[Considerações finais]{Considerações finais}

Durante o desenvolvimento da \textit{engine} e do jogo, em diversos momentos bastante tempo foi empregado tentando entender detalhes da especificação do \textit{hardware} do GBA, como a definição da paleta de cores dos \textit{backgrounds} e \textit{sprites} serem bastante diferentes, o \textit{overlap} entre \textit{charblocks} e \textit{screenblocks} na região de memória VRAM, os diferentes canais de áudio que serviam para propósitos bem diferentes, a impossibilidade de se utilizar uma ferramenta de depuração de código, como \texttt{gdb}\footnote{\textit{GNU Debugger}, disponível em \url{https://bit.ly/2r2Wzza}}, entre outros.

Além dos problemas relacionados ao \textit{hardware} do GBA, adaptar os recursos do jogo original de forma que pudessem ser carregados no GBA e testar o jogo no \textit{console} foram outras situações que demandaram boa parte do tempo do desenvolvimento do jogo. A dificuldade em testar o jogo no \textit{console} se deu principalmente pelo fato do cliente utilizado para realizar a transferência das ROM's possuir versão apenas para \textit{Windows XP}, sendo necessário utilizar uma máquina virtual apenas para esse processo, e pela sincronização entre o cliente e o dispositivo físico não ser constante, deixando de funcionar diversas vezes sem motivo aparente. Outro grande impedimento foi a conversão dos áudios para um formato adequado para o GBA. Após muita pesquisa, foram encontradas ferramentas que permitiam realizar tal conversão, mas que necessitavam que o \textit{sample rate} dos áudios fosse reduzido, ocasionando uma perda notável na qualidade final das músicas. 

Durante o processo de desenvolvimento do jogo, foram encontrados pontos de melhoria em certas tarefas/ações. Um dos pontos seria realizar uma melhor priorização das tarefas a serem executadas (por exemplo, o módulo de áudio foi o último a ser implementado). Outro ponto seria aumentar a frequência de realização de testes do jogo no \textit{console}. Isso ajudaria a validar o impacto das melhorias no \textit{hardware} real, já que o emulador não necessariamente reproduz à risca o comportamento do GBA.

Por fim, como resultado final do trabalho, a pergunta de pesquisa pôde ser respondida afirmativamente, significando que foi possível portar o jogo \textit{Traveling Will}, desenvolvido originalmente para PC, para o \textit{Nintendo Gameboy Advance}, no contexto de um trabalho de conclusão de curso, com performance e jogabilidade próximos da versão para computador.

\section{Trabalhos futuros}

Como sugestões de trabalhos futuros, têm-se: melhorar módulo de áudio para permitir carregar efeitos sonoros e pausar músicas durante a execução do jogo, implementar carregamento e utilização de fontes no jogo, adicionar elementos de HUD e seleção de fases, salvar o estado do jogo em memória e implementar um desfragmentador de memória na classe \texttt{MemoryManager}.
