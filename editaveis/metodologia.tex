\chapter[Metodologia]{Metodologia}

\section{Ferramentas de desenvolvimento}

  compilador

  devkitARM PRO

  emulador

  EZFlash II

  Nintendo DS

  Para a realização do porte do jogo para Game Boy Advance, são necessários dois ambientes principais: um ambiente de desenvolvimento onde seja possível implementar o jogo e exportar o binário executável para o console e um ambiente para testar o executável gerado, sendo esse físico ou emulado.

  \subsection{Ambiente de desenvolvimento}

    O jogo será reescrito utilizando a linguagem C++, na versão 2011, pois provê uma série de recursos e estruturas não presentes em C que facilitarão o desenvolvimento do jogo.

    O ambiente de desenvolvimento utilizado para a implementação do jogo consiste das seguintes ferramentas:

    \subsubsection{\textit{devkitPro} e \textit{devkitARM}}

      O devkitPro é uma organização que provê conjuntos de ferramentas para desenvolvimento de jogos em diversos consoles da Nintendo, como Nintendo GBA, Nintendo Wii, Nintendo Switch, dentre outros.

      Dentre esses conjuntos de ferramentas encontra-se o devkitARM, toolchain que contém o ambiente de desenvolvimento necessário para realizar a compilação do código escrito em C/C++ para a arquitetura de processadores ARM existente no GBA, citado no capítulo X.

    Para o ajuste das imagens do jogo para a resolução de tela do GBA, será utilizada a ferramenta de manipulação de imagens GIMP, versão 2.8.

  \subsection{Ambiente de teste}

    \subsubsection{Emulador}

      Para a realização de testes com os executáveis gerados pelo devkitARM está sendo utilizado um emulador de Game Boy Advance, chamado VisualBoyAdvance-M \footnote{colocar link do visualboyadvance-m}.

    \subsubsection{\textit{Console}}

      % verificar modelo do nintendo ds

      O console que está sendo utilizado como ambiente de testes real é um Nintendo DS Lite \footnote{inserir link do ds}, que possui um slot para cartuchos de Game Boy Advance. Neste trabalho está sendo utilizado um cartucho especial onde é possível escrever arquivos executáveis nele.

      Para a escrita dos arquivos executáveis neste cartucho é utilizado o dispositivo \textbf{EZFlash II} \footnote{inserir link do ez flash II}. Como essa é uma versão antiga do produto, é necessário instalar um cliente para upload dos arquivos para o cartucho. Este cliente só possui compatibilidade com \textit{\textbf{Windows XP}} \footnote{inserir link do windows xp}, fazendo com que seja necessário instalar um máquina virtual com o sistema operacional.

\section{Metodologia de desenvolvimento}

  O processo de reescrita do jogo para a plataforma \textit{Game Boy Advance} será dividido entre as seguintes tarefas que serão desenvolvidas concomitantemente:

  \begin{itemize}
    \item Estudar a documentação fornecida no \textit{GBATek}, documento que possui informações técnicas sobre o GBA.

    \item Desenvolver uma \textit{engine} com o objetivo de encapsular cada uma das principais funcionalidades necessárias para se desenvolver um jogo, como por exemplo desenhar \textit{sprites} na tela, trabalhar com colisões e atualizar os estados do jogo.

    \item Ajustar recursos como imagens e áudios para que possam ser carregados no GBA, levando em consideração que os recursos originais foram projetados para uma plataforma com mais memória e poder de processamento que os disponíveis no GBA.

    \item Desenvolver o jogo utilizando a engine desenvolvida e os recursos ajustados durante o projeto.
  \end{itemize}

  \subsection{Desenvolvimento da \textit{Engine}}

  botar esquemáticos

  módulo de vídeo (renderização de imagens)

  módulo de áudio

  módulo de física

  módulo de input

  \subsubsection{Módulo de vídeo}
    FIXME

    Um método para renderização de imagens. Ele deverá receber a imagem já no formato adequado para o GBA, a posição da tela em que ela deverá ser renderizada e as dimensões da imagem.

  \subsubsection{Módulo de áudio}

  \subsubsection{Módulo de física}

  \subsubsection{Módulo de \textit{input}}

  % \begin{itemize}
  % \item

  % \item Um método para atualização do estado dos itens do jogo. Esse método deverá lidar com os componentes de acordo com as regras do jogo. Por exemplo, se é esperado que um projétil se mova a uma velocidade x em determinado momento do jogo, esse método irá atualizar a posição do projétil continuamente, de acordo com a velocidade estabelecida.

  % \item Um método responsável por detectar colisões do jogo. Esse método irá receber dois ou mais objetos do jogo e deverá informar se tais objetos estão colidindo, sem lidar com as consequências da colisão.
  % \end{itemize}

\section{Ajuste de recursos do jogo}

  Os recursos como imagens e áudios do jogo precisarão ser editados para que possam ser carregados em memória e utilizados no GBA. No caso das imagens, por exemplo, poderão ser modificadas características como dimensões e cores a fim de diminuir seu tamanho para que possam, posteriormente, ser convertidas para um formato utilizável no GBA.
