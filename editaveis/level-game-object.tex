\subsection{Abstrações de níveis e objetos do jogo}

Além de implementar módulos relacionados a entrada (\textit{input}) e saída (vídeo, áudio, entre outros), a \textit{engine} também implementa abstrações para os níveis e para os objetos do jogo.

Os \textit{game objects} são a base para qualquer elemento utilizado no jogo. Um \textit{game object} possui uma posição \texttt{(x, y)} no espaço do jogo, possui velocidade horizontal e vertical, possui \textit{game objects} filhos (que formam uma hierarquia de \textit{game objects}) e tem como responsabilidades atualizar seu próprio estado, se desenhar na tela e atualizar os seus filhos.

Pelo fato de o mecanismo de renderização ser, de certa forma, automático (basta carregar as imagens na memória e especificar os metadados que elas já são renderizadas na tela), a função de se desenhar na tela não é utilizada. Porém, a atualização e controle dos filhos é importante para o comportamento correto desses objetos. O código \ref{lst:gameobj} apresenta a classe \texttt{GameObject}.

\begin{lstlisting}[label={lst:gameobj},caption={Classe \texttt{GameObject}}]
#ifndef GAME_OBJECT_H
#define GAME_OBJECT_H

#include <vector>
#include <stdint.h>

using std::vector;

class GameObject {
    protected:
        GameObject *m_parent;
        int m_x, m_y;
        int m_speed_x, m_speed_y;
        vector <GameObject *> m_children;

        GameObject *parent() const { return m_parent; }

        virtual void update_self(uint64_t dt) = 0;
        virtual void draw_self() = 0;

    public:
        void update(uint64_t dt);
        void draw();

        void set_parent(GameObject *obj) { m_parent = obj; }
        void add_child(GameObject *);
        void remove_child(GameObject *);
};

#endif
\end{lstlisting}

Com a definição e implementação dos \textit{game objects}, a caracterização da abstração dos níveis do jogo se torna bem simples. De forma geral, um nível é um \textit{game object} que, além de possuir todas as propriedades já explicitadas acima, possui uma variável que indica se o nível foi finalizado e outra que guarda o próximo nível a ser renderizado. O código \ref{lst:level} apresenta a implementação da classe \texttt{Level}.

\begin{lstlisting}[label={lst:level},caption={Classe \texttt{Level}}]
#ifndef LEVEL_H
#define LEVEL_H

#include "game_object.h"

#include <stdio.h>

class Level : public GameObject {
    protected:
        int m_x, m_y;
        bool m_done;
        int m_next;
};

#endif
\end{lstlisting}