\begin{resumo}

A massiva utilização de game engines no desenvolvimento de jogos atuais provê diversos benefícios aos desenvolvedores de jogos em termos de velocidade de desenvolvimento e quantidade de funcionalidades já implementadas. Entretanto quando é necessário desenvolver funcionalidades específicas, o uso de tais ferramentas acarreta na necessidade do emprego de workarounds, que podem prejudicar a performance geral do jogo. Esses problemas de performance são dificilmente notados em plataformas atuais, por possuírem recursos em abundância, porém podem se tornar um gargalo em plataformas antigas, que possuem memória e/ou armazenamento limitados, ou quando é necessário extrair o máximo de recursos de uma plataforma, como em jogos com gráficos robustos. A fim de testar o desenvolvimento de um jogo em uma plataforma com recursos limitados quando comparados a um PC, este trabalho tem como objetivo realizar o porte do jogo Traveling Will, desenvolvido originalmente para PC, para o videogame portátil Nintendo Game Boy Advance.

 \vspace{\onelineskip}

 \noindent
 \textbf{Palavras-chaves}: porte, jogos eletrônicos, jogos indie, nintendo,  game boy advance, performance.
\end{resumo}
