\section{Porte de jogos}

Na engenharia de software, porte é definido como ``mover um sistema entre ambientes ou plataformas'' \cite{frakes}. No contexto de jogos eletrônicos a definição é um pouco mais específica e, segundo Carreker \cite{carreker}, é o processo de converter código e outros recursos desenvolvidos para uma plataforma específica para outra plataforma, que, nesse caso, representa algum tipo de console ou computador.

Segundo Horna \cite{wawro}, o processo de porte de um jogo para uma plataforma específica consiste, geralmente, em três passos:

\begin{enumerate}
  \item Primeiramente deve-se conseguir executar o jogo na plataforma de destino (pelo menos compilar, com chamadas \textit{stub} \footnote{\textit{Stub is a dummy function that assists in testing part of a program. A stub has the same name and interface as a function that actually would be called by the part of the program being tested, but it is usually much simpler.} \cite{dale}} para a biblioteca de gráficos). Este processo tende a ser o mais difícil, pois há diversos problemas com bibliotecas específicas;
  \item Adicionar o suporte à biblioteca de gráficos da plataforma de destino, criando uma interface comum para todas as plataformas (de modo a manter o código mais manutenível);
  \item Após adicionar o suporte à bliblioteca de gráficos, é necessário otimizar a performance do jogo, principalmente otimizações que levam em conta a performance da CPU, que podem afetar a execução do jogo de maneiras difíceis de antecipar.
\end{enumerate}

% - o que é porte?
% - características de porte
%   - porte de código
%   - porte de recursos do jogo
% - vantagens de porte
%   -
% - desvantagens de porte
%   - retrabalho
%   - dificuldade de adaptação do jogo para a plataforma de destino